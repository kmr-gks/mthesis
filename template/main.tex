\documentclass[11pt]{ist-thesis}
\pagestyle{bachelorthesis}  % 卒論
%\pagestyle{ebachelorthesis}  % Graduation Thesis
%\pagestyle{masterthesis}  % 修論(目安;38ページ以上)
%\pagestyle{emasterthesis}  % Master Thesis
%\pagestyle{draft}  % 未完成ドラフト
%\pagestyle{edraft}  % Draft

\title{論文のタイトル\\タイトル2行目}
\author{論文 太郎}
\supervisor{?? 教授}
%\supervisor{Professor ??}
\deadline{令和?年?月?日} % 正規の提出期日
%\deadline{February Xth, XXXX}

\begin{document}

\titlepage

% タイトル中に改行が含まれる場合は,改行を取り除いたタイトルを再定義
\title{論文タイトル}

\abstract{
ここに概要を書く
}

\keyword{
  \hspace{7mm} 論文のキーワードその1\\
  \hspace{7mm} 論文のキーワードその2\\
  \hspace{7mm} 論文のキーワードその3\\
}

% 目次を出力
\toc{}

% 本文開始
\section{Introduction}

\subsection{研究の背景}
オープンソースソフトウェア(OSS)のエコシステムは、現代社会におけるデジタル基盤として極めて重要な役割を担っている。世界中のソフトウェア開発、企業活動、および学術研究の多くが、これらOSSプロジェクトに直接的あるいは間接的に依存している。

しかし、多くのOSSプロジェクトの実態は、少数の「コア開発者」による献身的な貢献に強く依存していることが報告されている。大規模なプロジェクトであっても、コードの根幹に関わる主要な変更を担う開発者は極めて限定的であり、彼らの離脱や燃え尽き(Burnout)は、プロジェクトの停滞や終了に直結する構造的な脆弱性を孕んでいる。

この脆弱性は、Heartbleed(OpenSSL)やLog4Shell(Log4j)といった深刻なセキュリティ脆弱性の露呈を通じて、社会的に大きな注目を集めた。世界中のインフラを支えるソフトウェアが、実質的に一握りのボランティアによって維持されているという事実は、現代のデジタルエコシステムにおける「持続可能性(Sustainability)」の欠如を浮き彫りにした。

\subsection{研究の動機と課題}
こうしたOSSの持続可能性に関する課題を解決するため、近年では Open Collective などのファンディングプラットフォームを通じて、プロジェクトが直接的に寄付金などの金銭的支援を募る仕組みが普及しつつある。これにより、従来はボランティアベースで行われていた開発活動に対し、外部から資金を供給し、プロジェクトの安定的な運営を支援することが可能となった。

理論上、OSSはボランティアによる自発的な貢献によって維持されるとされる。しかし、もし特定のプロジェクトが継続的な保守・開発のために外部からの資金援助に強く依存している場合、その資金の途絶はプロジェクトの寿命(Longevity)や、ひいてはそれらに依存するエコシステム全体の健全性に深刻な影響を及ぼす可能性がある。

現在、多くのプロジェクトが資金を獲得しているものの、提供された「資金」が具体的に「ソフトウェア開発活動」にどのような影響を及ぼしているか、その相関関係や因果関係については、定量的には十分に解明されていない。

\subsection{研究の目的}
本研究の目的は、外部からのファンディングがOSSプロジェクトのソフトウェア開発に与える影響とその関係性を、データに基づいて調査することである。具体的には、以下の3つのリサーチクエスチョン(RQ)に取り組む。

\begin{enumerate}
    \item \textbf{資金提供の実態調査}:どのような背景や目的で、どの程度の規模の資金がOSSプロジェクトに提供されているのかを明らかにする。
    \item \textbf{資金使途の分析}:プロジェクトが受け取った資金が、エンジニアリング、マーケティング、あるいは運営維持など、どのような目的に支出されているのかを調査する。
    \item \textbf{開発活動への影響の定量的測定}:資金の獲得および支出が、GitHub上の開発アクティビティ(プルリクエスト数、コミット数、アクティブ開発者数など)とどのように相関しているかを定量的に評価する。
\end{enumerate}

\subsection{研究の手法(概要)}
本研究では、Open Collective APIを使用して、対象となるOSSプロジェクトの全取引履歴を収集する。これには、支出目的、金額、日付などの詳細なメタデータが含まれる。

次に、収集した資金データとGitHubのコントリビューション履歴を紐付け、資金獲得の前後、あるいは特定の支出(例:エンジニアリングへの支払い)の前後で、開発アクティビティにどのような変化が生じているかを比較分析する。これにより、資金の投入が直接的に活発なメンテナンスや開発の促進に寄与しているのかを検証する。

\subsection{研究の意義}
本研究の対象となるプロジェクトは、世界中の企業、組織、研究者が利用するクリティカルなOSSインフラストラクチャである。これらのプロジェクトが実際に寄付金に依存して維持されている実態が明らかになれば、それはエコシステム全体の潜在的なリスクを示す重要な知見となる。

本研究の結果は、OSSの持続可能性を模索する開発者コミュニティだけでなく、OSSに依存する企業や、支援の枠組みを設計するファンディングプラットフォーム運営者にとって、資金提供の有効性とプロジェクトの健全性を客観的に判断するための重要な指針を提供するものである。



1章本文

	•	どのようなプロジェクトがどの程度の資金を受け取っているのか
	•	受け取った資金がどのような目的で使用されているのか
	•	外部資金の受領が開発活動の活発さと関連しているのか

といった点は明らかになっていない。

1.1節本文
\subsubsection{1.1.1項タイトル}
1.1.1項本文


参考文献を参照する\cite{KamiyaTSE2002}

\section{Background and Related Work}

\section{Research Questions}

\acknowledgement{
ここに謝辞を書く
}

% 参考文献を出力
\bibliographystyle{jplain}
\bibliography{reference}

\end{document}
