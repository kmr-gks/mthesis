\documentclass[11pt]{ist-thesis}
\pagestyle{bachelorthesis}  % 卒論
%\pagestyle{ebachelorthesis}  % Graduation Thesis
%\pagestyle{masterthesis}  % 修論(目安;38ページ以上)
%\pagestyle{emasterthesis}  % Master Thesis
%\pagestyle{draft}  % 未完成ドラフト
%\pagestyle{edraft}  % Draft

\title{論文のタイトル\\タイトル2行目}
\author{論文 太郎}
\supervisor{?? 教授}
%\supervisor{Professor ??}
\deadline{令和?年?月?日} % 正規の提出期日
%\deadline{February Xth, XXXX}

\begin{document}

\titlepage

% タイトル中に改行が含まれる場合は,改行を取り除いたタイトルを再定義
\title{論文タイトル}

\abstract{
ここに概要を書く
}

\keyword{
  \hspace{7mm} 論文のキーワードその1\\
  \hspace{7mm} 論文のキーワードその2\\
  \hspace{7mm} 論文のキーワードその3\\
}

% 目次を出力
\toc{}

% 本文開始
\section{Introduction}
現代社会の多くのサービスやシステムは、オープンソースソフトウェア(OSS)によって支えられている。例えば、Linux、OpenSSL、Chromium、NumPy などは、世界中の基盤ソフトウェアとして広く利用されている。これらのOSSが安定的に機能し続けるためには、継続的な開発と保守が不可欠である。

しかし、既存研究によれば、多くのOSSプロジェクトは少数の「コア開発者」に依存していることが報告されている。大規模プロジェクトであっても、実際に主要な変更を担う開発者は限られており、彼らの離脱はプロジェクトの停滞や終了につながる可能性がある。この問題は、Heartbleed や Log4Shell のような深刻なセキュリティ事例を通じて社会的にも注目された。

近年、この持続可能性問題を緩和する手段として、OpenCollective や GitHub Sponsors といった資金提供プラットフォームが登場した。これらのプラットフォームは、企業や個人がOSSプロジェクトに対して金銭的支援を行う仕組みを提供している。

しかしながら、OSSへの寄付が実際に開発活動にどのような影響を与えているのか、また受け取った資金がどのように使用されているのかについては、十分な実証的分析が行われていない。

本研究では、OpenCollective上のOSSプロジェクトを対象に、資金の流入および支出データをマイニングし、さらに関連するGitHubリポジトリから開発活動データ(コミット履歴など)を収集する。これらのデータを統合的に分析することで、OSSへの資金提供が開発活動に与える影響、および資金の使用構造を明らかにすることを目的とする。本研究は、OSSの持続可能性に関する理解を深め、より健全なエコシステム構築への示唆を提供する。

1章本文
\subsection{1.1節タイトル}
1.1節本文
\subsubsection{1.1.1項タイトル}
1.1.1項本文

参考文献を参照する\cite{KamiyaTSE2002}

\section{Background and Related Work}

\section{Research Questions}

\acknowledgement{
ここに謝辞を書く
}

% 参考文献を出力
\bibliographystyle{jplain}
\bibliography{reference}

\end{document}
