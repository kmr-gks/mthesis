\documentclass[11pt]{ist-thesis}
\pagestyle{bachelorthesis}  % 卒論
%\pagestyle{ebachelorthesis}  % Graduation Thesis
%\pagestyle{masterthesis}  % 修論(目安;38ページ以上)
%\pagestyle{emasterthesis}  % Master Thesis
%\pagestyle{draft}  % 未完成ドラフト
%\pagestyle{edraft}  % Draft

\title{論文のタイトル\\タイトル2行目}
\author{論文 太郎}
\supervisor{?? 教授}
%\supervisor{Professor ??}
\deadline{令和?年?月?日} % 正規の提出期日
%\deadline{February Xth, XXXX}

\begin{document}

\titlepage{}

% タイトル中に改行が含まれる場合は,改行を取り除いたタイトルを再定義
\title{論文タイトル}

\abstract{
ここに概要を書く
}

\keyword{
  \hspace{7mm} 論文のキーワードその1\\
  \hspace{7mm} 論文のキーワードその2\\
  \hspace{7mm} 論文のキーワードその3\\
}

% 目次を出力
\toc{}

% 本文開始
\section{Introduction}

\subsection{Research Background}
Open-source software (OSS) ecosystems have become an indispensable digital foundation for modern society. A vast majority of software development, corporate operations, and academic research worldwide depend directly or indirectly on these OSS projects.

However, existing literature reports that many OSS projects rely heavily on the dedicated contributions of a small number of ``core developers''. Even in large-scale projects, the developers responsible for the most critical parts of the codebase are often extremely limited in number. This structure entails a inherent vulnerability: the departure or burnout of these key individuals can lead directly to the stagnation or termination of the project.

This structural vulnerability gained significant public attention following severe security incidents such as Heartbleed (OpenSSL) and Log4Shell (Log4j). The fact that software underpinning global infrastructure is maintained by a handful of volunteers has highlighted a critical lack of ``sustainability'' in the modern digital ecosystem.

\subsection{Motivation and Problem Statement}
To address these sustainability challenges, funding platforms such as Open Collective have emerged, allowing projects to solicit financial support directly. These platforms provide a mechanism to inject external capital into activities that were previously performed on a purely volunteer basis, thereby supporting stable project management.

In theory, OSS projects are maintained by voluntary contributions. However, if these projects increasingly depend on external funding for continuous maintenance and development, the cessation of such funds could jeopardize the longevity of the projects and, by extension, the health of the entire ecosystem that relies on them.

While many projects are now receiving significant financial contributions, the specific impact of these ``funds'' on actual ``software development activities'' remains poorly understood. Specifically, the correlation and causal relationships between funding and development productivity have not yet been sufficiently quantified.

\subsection{Research Objectives}
The objective of this research is to investigate the relationship between external funding and software development activities in OSS projects based on empirical data. Specifically, we address the following three Research Questions (RQs):

\begin{enumerate}
    \item \textbf{RQ1: Investigation of Financial Contributions.} What are the motivations and scales of financial contributions made to OSS projects?
    \item \textbf{RQ2: Analysis of Funding Expenditures.} For what purposes (e.g., engineering, marketing, or operations) do projects spend the funds they receive?
    \item \textbf{RQ3: Quantitative Measurement of Impact on Development.} How do the acquisition and expenditure of funds correlate with GitHub development activities, such as the number of pull requests, commits, and active developers?
\end{enumerate}

\subsection{Methodology Overview}
In this study, we utilize the Open Collective API to collect the complete transaction histories of target OSS projects. This dataset includes granular metadata such as expenditure purposes, transaction amounts, and timestamps.

Next, we correlate this financial data with GitHub contribution histories to analyze changes in development activity before and after the acquisition of funds or specific expenditures (e.g., payments for engineering tasks). Through this approach, we verify whether the injection of capital directly contributes to more active maintenance and development.

\subsection{Significance of the Research}
The projects targeted in this study are critical OSS infrastructures used by companies, organizations, and researchers worldwide. If it is revealed that these projects rely heavily on donations for their maintenance, it will provide crucial insights into the potential risks and vulnerabilities of the global software ecosystem.

The findings of this research will provide essential guidelines for developer communities seeking sustainability, companies dependent on OSS, and funding platform operators to evaluate the effectiveness of financial support and the overall health of projects.

\section{Background and Related Work}

\acknowledgement{
ここに謝辞を書く
参考文献を参照する\cite{KamiyaTSE2002}
}

% 参考文献を出力
\bibliographystyle{jplain}
\bibliography{reference}

\end{document}
