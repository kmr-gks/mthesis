\documentclass[11pt]{ist-thesis}
\pagestyle{bachelorthesis}  % 卒論
%\pagestyle{ebachelorthesis}  % Graduation Thesis
%\pagestyle{masterthesis}  % 修論(目安;38ページ以上)
%\pagestyle{emasterthesis}  % Master Thesis
%\pagestyle{draft}  % 未完成ドラフト
%\pagestyle{edraft}  % Draft

\title{論文のタイトル\\タイトル2行目}
\author{論文 太郎}
\supervisor{?? 教授}
%\supervisor{Professor ??}
\deadline{令和?年?月?日} % 正規の提出期日
%\deadline{February Xth, XXXX}

\begin{document}

\titlepage{}

% タイトル中に改行が含まれる場合は,改行を取り除いたタイトルを再定義
\title{論文タイトル}

\abstract{
ここに概要を書く
}

\keyword{
  \hspace{7mm} 論文のキーワードその1\\
  \hspace{7mm} 論文のキーワードその2\\
  \hspace{7mm} 論文のキーワードその3\\
}

% 目次を出力
\toc{}

% 本文開始
\section{Introduction}

\subsection{Research Background}
Open-source software (OSS) ecosystems have become an indispensable digital foundation for modern society. A vast majority of software development, corporate operations, and academic research worldwide depend directly or indirectly on these OSS projects.

However, existing literature reports that many OSS projects rely heavily on the dedicated contributions of a small number of ``core developers''\cite{sample}. Core developers are defined as a key group of contributors who possess deep technical knowledge of the project and hold the authority to make significant changes to the source code. Even in large-scale projects, the developers responsible for the most critical parts of the codebase are often extremely limited in number\cite{sample}. This structure entails an inherent vulnerability: the departure or burnout of these key individuals can lead directly to the stagnation or termination of the project.

This structural vulnerability gained significant public attention following severe security incidents such as Heartbleed (OpenSSL)\cite{sample} and Log4Shell (Log4j)\cite{sample}. The fact that software underpinning global infrastructure is maintained by a handful of volunteers has highlighted a critical lack of sustainability in the modern digital ecosystem\cite{sample}.

\subsection{Motivation and Problem Statement}
To address these sustainability challenges, funding platforms such as Open Collective\cite{sample} and GitHub Sponsors\cite{sample} have emerged, allowing projects to solicit financial support directly. These platforms provide a mechanism to inject external capital into activities that were previously performed on a purely volunteer basis, thereby supporting stable project management.

In theory, OSS projects are maintained by voluntary contributions. However, if these projects increasingly depend on external funding for continuous maintenance and development, the cessation of such funds could jeopardize the longevity of the projects and, by extension, the health of the entire ecosystem that relies on them.

While many projects are now receiving significant financial contributions, the specific impact of these funds on actual software development activities remains poorly understood. Specifically, the correlation and causal relationships between funding and development productivity have not yet been sufficiently quantified.

\subsection{Research Objectives}
The objective of this research is to investigate the relationship between external funding and software development activities in OSS projects based on empirical data. Specifically, we address the following three Research Questions:

\begin{enumerate}
    \item \textbf{RQ1: Investigation of Financial Contributions.} What are the motivations and scales of financial contributions made to OSS projects?
    \item \textbf{RQ2: Analysis of Funding Expenditures.} For what purposes (e.g., engineering, marketing, or operations) do projects spend the funds they receive?
    \item \textbf{RQ3: Quantitative Measurement of Impact on Development.} How do the acquisition and expenditure of funds correlate with GitHub development activities, such as the number of pull requests, commits, and active developers?
\end{enumerate}

\subsection{Methodology Overview}
In this study, we utilize the Open Collective API\cite{sample} to collect the complete transaction histories of target OSS projects. This dataset includes granular metadata such as expenditure purposes, transaction amounts, and timestamps.

Next, we correlate this financial data with GitHub contribution histories to analyze changes in development activity before and after the acquisition of funds or specific expenditures (e.g., payments for engineering tasks). Through this approach, we verify whether the injection of capital directly contributes to more active maintenance and development.

\subsection{Significance of the Research}
The projects targeted in this study are critical OSS infrastructures used by companies, organizations, and researchers worldwide. If it is revealed that these projects rely heavily on donations for their maintenance, it will provide crucial insights into the potential risks and vulnerabilities of the global software ecosystem.

The findings of this research will provide essential guidelines for developer communities seeking sustainability, companies dependent on OSS, and funding platform operators to evaluate the effectiveness of financial support and the overall health of projects.

\section{Background and Related Work}

\subsection{OSS Sustainability and Core Developers}

OSSプロジェクトの持続可能性(Sustainability)は、少数の「コア開発者」への過度な依存という構造的な問題と密接に関連している。先行研究によれば、OSSのコードベースに関する深い知識を持ち、変更を承認する権限を持つ開発者は極めて限定的である。

この知識の集中度を測定する指標として「Truck Factor (TF)」(またはBus Factor)が提案されている。TFは、プロジェクトが機能を停止(あるいは深刻なトラブルに陥る)までに、何人の開発者が不慮の事故や離脱に見舞われる必要があるかを示す定量的指標である 。GitHub上の133の人気プロジェクトを対象とした調査では、プロジェクトの65%においてTFが2以下であることが明らかになっており、極めて少数の個人に依存している実態が示されている 。また、36,000以上のプロジェクトを対象とした大規模調査では、89%のプロジェクトが少なくとも一度はコア開発チームの離脱を経験しており、そのうち70%はプロジェクト開始から3年以内に発生している 。さらに、一度コア開発者が離脱したプロジェクトが新たなコア開発者を獲得し生存できる確率は、わずか27%に留まる 。
+3

こうした少人数への依存がもたらすリスクは、社会的なセキュリティ事件を通じて顕在化してきた。


Log4j (Log4Shell): 世界中の企業が利用するログ収集ライブラリでありながら、脆弱性発覚当時、実質的に少数のメンテナーが無報酬のボランティアとして、1日22時間にも及ぶ対応を強いられる事態となった 。


OpenSSL (Heartbleed): 世界のデジタルインフラを支える暗号化ソフトウェアでありながら、脆弱性が露呈する以前、その活動を支援する財団への寄付金は年間わずか2,000ドル程度に過ぎず、フルタイムで従事できる開発者は1名のみであった 。
+1

これらの事例は、重要なOSSインフラが「ボランティア頼み」の不安定な基盤の上に成り立っていることを示しており、コア開発者の負担軽減と持続可能性確保のための金銭的支援(Funding)の必要性が議論されるようになった 。

\subsection{Funding in the OSS Ecosystem}

近年、Open Collectiveのようなプラットフォームの登場により、OSSプロジェクトが直接的に資金を募り、その収支を透明化できる環境が整いつつある 。

Open Collectiveを利用するGitHubプロジェクトの分析によれば、寄付者の構成は個人(Individual)と企業(Corporate)に分けられる。企業は1回あたりの寄付額が大きい傾向にあるが、プロジェクト全体としての総寄付額では個人の寄付が企業のそれを上回るケースも多く、個人寄付者の重要性が示唆されている 。また、受け取った資金の使途については、エンジニアリング活動(新機能の実装やバグ修正)だけでなく、約54%が非エンジニアリング活動(Webサービス、マーケティング、旅費など)に支出されていることが報告されている 。
+1

しかし、これらの「資金の流入」や「支出」が、実際の「ソフトウェア開発アクティビティ」にどのような定量的影響を及ぼしているかについては、依然として研究が不十分である。特定の論文では、企業の寄付額とプロジェクトのメンテナンス活動レベルに正の相関があることが示されているが、支出の種類(エンジニアリングかそれ以外か)が開発効率にどう寄与するかまでは踏み込んでいない 。

\subsection{Empirical Studies on Development Activity}

本研究では、外部資金提供の影響を測定するために、ソフトウェア工学において確立されたメトリクスを用いる。
先行研究では、開発の「活発さ」や「コミュニティの反応」を測定するために、以下の指標が広く用いられている 。

リポジトリ・メトリクス: コミット数、プルリクエスト(PR)数、マージされたPR数。

コミュニティ・メトリクス: スター数、フォーク数、アクティブなコントリビューター数。

例えば、Hacker NewsなどのSNSでの露出(プロモーション)が開発アクティビティに与える影響を調査した研究では、投稿後の5ヶ月間にわたって、スター数、フォーク数、コントリビューター数が有意に増加することが示されている 。本研究においても、これらの指標を「資金獲得(Funding event)」や「支出(Expenditure)」の前後で比較分析する手法を採用する。これにより、資金の投入が直接的に活発なメンテナンスや開発の促進に寄与しているのか、あるいは単なるプロジェクトの維持に留まっているのかを検証する。

\acknowledgement{
ここに謝辞を書く
参考文献を参照する\cite{KamiyaTSE2002}
}

% 参考文献を出力
\bibliographystyle{jplain}
\bibliography{reference}

\end{document}
